\section{Решение нелинейных уравнений и систем нелинейных уравнений}

\subsection{Решение нелинейных уравнений}

\subsubsection{Постановка задачи}
Реализовать методы простой итерации и Ньютона решения нелинейных уравнений в виде программ, задавая в качестве входных данных точность вычислений. С использованием разработанного программного обеспечения найти положительный корень нелинейного уравнения (начальное приближение определить графически). Проанализировать зависимость погрешности вычислений от количества итераций.

{\bfseries Вариант:}
\begin{equation*}
\ln{x+2} - x^2 = 0
\end{equation*}

\subsubsection{Результаты работы}
\begin{alltt}
$ ./lab2_1
0.001
Метод простой итерации:
Корень: -0.58782084
Количество итераций: 49
Корень: 1.05723005
Количество итераций: 11

Метод Ньютона:
Корень: -0.58760883
Количество итераций: 6
Корень: 1.05710366
Количество итераций: 4

$ ./lab2_1
0.00000001
Метод простой итерации:
Корень: -0.58760883
Количество итераций: 127
Корень: 1.05710355
Количество итераций: 29

Метод Ньютона:
Корень: -0.58760883
Количество итераций: 7
Корень: 1.05710355
Количество итераций: 6
\end{alltt}

\subsubsection{Исходный код}
\lstinputlisting{../include/nonlinear/solvers.hpp}
\pagebreak

\subsection{Решение систем нелинейных уравнений}

\subsubsection{Постановка задачи}
Реализовать методы простой итерации и Ньютона решения систем нелинейных уравнений в виде программного кода, задавая в качестве входных данных точность вычислений. С использованием разработанного программного обеспечения решить систему нелинейных уравнений (при наличии нескольких решений найти то из них, в котором значения неизвестных являются положительными); начальное приближение определить графически. Проанализировать зависимость погрешности вычислений от количества итераций.

{\bfseries Вариант:}
\begin{equation*}
\begin{cases}
(x_1^2 + 9) x_2 - 27 = 0\\
(x_1 - 1.5)^2 + (x_2 - 1.5)^2 - 9 = 0\\
\end{cases}
\end{equation*}

\subsubsection{Результаты работы}
\begin{alltt}
$ ./lab2_2
0.001 1
Метод простой итерации:
Решение: -1.32480839 2.51043055
Количество итераций: 7
Решение: 4.44697542 0.93831944
Количество итераций: 9

Метод Ньютона:
Решение: -1.32469087 2.51050556
Количество итераций: 4
Решение: 4.44694707 0.93830348
Количество итераций: 4

$ ./lab2_2
0.00000001 1
Метод простой итерации:
Решение: -1.32469087 2.51050556
Количество итераций: 17
Решение: 4.44694707 0.93830348
Количество итераций: 20

Метод Ньютона:
Решение: -1.32469087 2.51050556
Количество итераций: 5
Решение: 4.44694707 0.93830348
Количество итераций: 5
\end{alltt}

\subsubsection{Исходный код}
\lstinputlisting{../include/nonlinear/system_solvers.hpp}
\pagebreak