\section{Численные методы линейной алгебры}

\subsection{LU-разложение матриц}

\subsubsection{Постановка задачи}
Реализовать алгоритм LU - разложения матриц (с выбором главного элемента) в виде программы. Используя разработанное программное обеспечение, решить систему линейных алгебраических уравнений (СЛАУ). Для матрицы СЛАУ вычислить определитель и обратную матрицу.

{\bfseries Вариант:}
\begin{equation*}
\begin{cases}
2x_1 + 7x_2 - 8x_3 + 6x_4 = -39\\
4x_1 + 4x_2 - 7x_3 - 7x_4 = 41\\
-x_1 - 4x_2 + 6x_3 + 3x_4 = 4\\
9x_1 - 7x_2 - 2x_3 - 8x_4 = 113\\
\end{cases}
\end{equation*}

\subsubsection{Результаты работы}
\begin{alltt}
$ cat tests/1/2.txt
4
2 7 -8 6
4 4 0 -7
-1 -3 6 3
9 -7 -2 -8
-39 41 4 113

$ ./lab1_1 < tests/1/2.txt
Решение системы:
   8.000   -3.000    2.000   -3.000
Обратная матрица системы:
   0.102    0.072    0.161    0.074
   0.040    0.111    0.035   -0.054
  -0.004    0.087    0.155   -0.021
   0.081   -0.038    0.112    0.011

Определитель матрицы системы:
-4924.000
\end{alltt}
\pagebreak

\subsubsection{Исходный код}
\lstinputlisting{../include/linear/lup.hpp}
\pagebreak

\subsection{Метод прогонки}

\subsubsection{Постановка задачи}
Реализовать метод прогонки в виде программы, задавая в качестве входных данных ненулевые элементы матрицы системы и вектор правых частей. Используя разработанное программное обеспечение, решить СЛАУ с трехдиагональной матрицей.

{\bfseries Вариант:}
\begin{equation*}
\begin{cases}
10x_1 + 5x_2 = -120\\
3x_1 + 10x_2 - 2x_3 = -91\\
2x_2 - 9x_3 - 5x_4 = 5\\
5x_3 + 16x_4 - 4x_5 = -74\\
-8x_4 + 16x_5 = -56\\
\end{cases}
\end{equation*}

\subsubsection{Результаты работы}
\begin{alltt}
$ cat tests/2/2.txt
5
10 5
3 10 -2
2 -9 -5
5 16 -4
-8 16
-120 -91 5 -74 -56

$ ./lab1_2 < tests/2/2.txt
Решение системы:
  -9.000   -6.000    2.000   -7.000   -7.000
\end{alltt}
\pagebreak

\subsubsection{Исходный код}
\lstinputlisting{../include/linear/tridiagonal_matrix.hpp}
\pagebreak

\subsection{Итерационные методы решения СЛАУ}

\subsubsection{Постановка задачи}
Реализовать метод простых итераций и метод Зейделя в виде программ, задавая в качестве входных данных матрицу системы, вектор правых частей и точность вычислений. Используя разработанное программное обеспечение, решить СЛАУ. Проанализировать количество итераций, необходимое для достижения заданной точности.

{\bfseries Вариант:}
\begin{equation*}
\begin{cases}
24x_1 +2x_2 +4x_3 -9x_4 = -9\\
-6x_1 -27x_2 -8x_3 -6x_4 = -76\\
-4x_1 +8x_2 +19x_3 +6x_4 = -79\\
4x_1 +5x_2 -3x_3 -13x_4 = -70\\
\end{cases}
\end{equation*}

\subsubsection{Результаты работы}
\begin{alltt}
$ cat tests/3/2.txt
4
24 2 4 -9
-6 -27 -8 -6
-4 8 19 6
4 5 -3 -13
-9 -76 -79 -70
0.0001

$ ./lab1_3 < tests/3/2.txt
Решение методом Якоби:
3.999999 1.999999 -6.999998 8.999998
Количество итераций: 20

Решение методом Зейделя:
4.000000 2.000000 -7.000000 9.000000
Количество итераций: 10
\end{alltt}
\pagebreak

\subsubsection{Исходный код}
\lstinputlisting{../include/linear/iteration_solvers.hpp}
\pagebreak

\subsection{Метод вращений}

\subsubsection{Постановка задачи}
Реализовать метод вращений в виде программы, задавая в качестве входных данных матрицу и точность вычислений. Используя разработанное программное обеспечение, найти собственные значения и собственные векторы симметрических матриц. Проанализировать зависимость погрешности вычислений от числа итераций.

{\bfseries Вариант:}
\begin{equation*}
\begin{pmatrix}
-9& 7& 5\\
7& 8& 9\\
5& 9& 8\\
\end{pmatrix}
\end{equation*}

\subsubsection{Результаты работы}
\begin{alltt}
$ cat tests/4/2.txt
3
-9 7 5
7 8 9
5 9 8
0.0001

$ ./lab1_4 < tests/4/2.txt
Собственные значения:
  19.532   -0.836  -11.696

Матрица собственных векторов:
   0.286   -0.115    0.951
   0.691   -0.663   -0.288
   0.664    0.740   -0.110

Количество итераций: 6
\end{alltt}
\pagebreak

\subsubsection{Исходный код}
\lstinputlisting{../include/linear/rotation_method.hpp}
\pagebreak

\subsection{QR алгоритм}

\subsubsection{Постановка задачи}
Реализовать алгоритм QR – разложения матриц в виде программы. На его основе разработать программу, реализующую QR – алгоритм решения полной проблемы собственных значений произвольных матриц, задавая в качестве входных данных матрицу и точность вычислений. С использованием разработанного программного обеспечения найти собственные значения матрицы.

{\bfseries Вариант:}
\begin{equation*}
\begin{pmatrix}
-6& -4& 0\\
-7& 6& -7\\
-2& -6& -7\\
\end{pmatrix}
\end{equation*}

\subsubsection{Результаты работы}
\begin{alltt}
$ cat tests/5/2.txt
3
-6 -4 0
-7 6 -7
-2 -6 -7
0.00001

$ ./lab1_5 < tests/5/2.txt
Собственные значения:
-11.3395 10.0120 -5.6725
Количество итераций: 112
\end{alltt}
\pagebreak

\subsubsection{Исходный код}
\lstinputlisting[extendedchars=\true]{../include/linear/qr_decomposition.hpp}
\pagebreak