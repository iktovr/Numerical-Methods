\section{Описание}

\subsection*{Экспоненциальный сплайн}

Кубический сплайн имеет прямую аналогию в теории упругости, а именно описывает поведение гибкого стрежня, закрепленного в некоторых точках. С этой точки зрения, кубический сплайн $s$ на каждом интервале $[x_i, x_{i+1}]$, $i = 1, \ldots, N$ является решением следующей краевой задачи:
$$[D^4]s = 0,\ s(x_i) = f_i,\ s(x_{i+1}) = f_{i+1},\ s''(x_i) = s''_i,\ s''(x_{i+1}) = s''_{i+1}$$
Где $s''_i$ и $s''_{i+1}$ подобраны так, чтобы $s \in C^2[a,b]$, и получаются как решение системы с трехдиагональной матрицей.

Полученный таким образом сплайн имеет склонность к образованию точек перегиба независимо от того, соответствует ли это исходным данным или нет. Точки перегиба, возникающие вследствие нежелательных колебаний интерполяционной кривой, называют ложными точками перегиба.

Во избежании таких точек, наложим постоянное \enquote{натяжение} на интервалы, где они появляется. Определим экспоненциальный сплайн $\tau$ как решение совокупности краевых задач на интервалах $[x_i, x_{i+1}]$, $i = 1, \ldots, N$ вида:
$$[D^4 - p^2_i D^2]\tau = 0,\ \tau(x_i) = f_i,\ \tau(x_{i+1}) = f_{i+1},\ \tau''(x_i) = \tau''_i,\ \tau''(x_{i+1}) = \tau''_{i+1}$$
Где $p_i$, $i = 1, \ldots, N$ --- параметр натяжения.

При этом имеется два предельных случая:
\begin{enumerate}
	\item $p_i \rightarrow 0 \ \Rightarrow \ [D^4 - p^2_i D^2]\tau = 0 \ \Rightarrow \ [D^4]\tau = 0$. Экспоненциальный сплайн вырождается в кубический сплайн.
	\item $p_i \rightarrow \infty \ \Rightarrow \ [D^4 - p^2_i D^2]\tau = 0 \ \Rightarrow \ [(1/p^2_i)D^4 - D^2]\tau = 0 \ \Rightarrow \ [D^2]\tau = 0$. Экспоненциальный сплайн вырождается в сплайн первого порядка --- ломаную линию.
\end{enumerate}

На интервале $[x_i, x_{i+1}]$, $i = 1, \ldots, N$ экспоненциальный сплайн задается формулой
$$\tau(x) = \frac{1}{p_i^2 S_i}\Big[\tau''_i \sh(p_i(x_{i+1} - x)) + \tau''_{i+1} \sh(p_i(x - x_i))\Big] + 
\left(f_i - \frac{\tau''_i}{p_i^2}\right)\frac{x_{i+1}-x}{h_i} + 
\left(f_{i+1} - \frac{\tau''_{i+1}}{p_i^2}\right)\frac{x-x_i}{h_i}$$

Где $f_i$, $i = 1, \ldots, N+1$ --- значения функции в узлах интерполяции;\\
$h_i = x_{i+1} - x_i$,  $S_i = \sinh(p_i h_i)$, $i = 1, \ldots, N$;\\
$\tau''_i$ определяются решением системы уравнений с трехдиагональной матрицей:
$$
\begin{cases}
d_1 \tau''_1 = b_1 \\
e_{i-1} \tau''_{i-1} + (d_{i-1} + d_i)\tau''_i + e_i \tau''_{i+1} = b_i,\ (i = 2, \ldots, N) \\
d_N \tau''_{N+1} = b_{N+1}
\end{cases}
$$
Где:
\begin{align*} 
e_i &= \left(\frac{1}{h_i} - \frac{p_i}{S_i}\right) /p^2_i & \\
d_i &= \left(p_i \frac{C_i}{S_i} - \frac{1}{h_i}\right)/p^2_i & i = 1, \ldots, N\\
C_i &= \cosh(p_i h_i) & \\
\\
b_1 &= b_{N+1} = 0 & \\
b_i &= \frac{f_{i+1} - f_i}{h_i} - \frac{f_i - f_{i-1}}{h_{i-1}} & i = 2, \ldots, N \\
\end{align*} 

Данная система соответствует сплайну с естественными граничными условиями ($\tau''_1 = \tau''_{N+1} = 0$).

\subsection*{Выбор параметров натяжения}

Отдельный интерес представляет задача выбора параметра натяжения для каждого из интервалов интерполяции, достаточного для устранения ложных точек перегиба.
Ложные точки перегиба однозначно устраняются при \enquote{достаточно больших} значениях параметра, но это также приводит к появлению областей большой кривизны вблизи узлов.

Отсутствие точек перегиба на интервале $[x_i, x_{i+1}]$ означает постоянство знака второй производной $\tau''$. Тогда условия $\tau''_ib_i > 0$, $\tau''_{i+1}b_{i+1} > 0$ являются необходимыми и достаточными для этого. Таким образом появляется возможность итеративно изменять параметр $p_i$ до достижения условия $\tau''_ib_i > 0$ $i = 1, \ldots, N+1$.

Так как рассматривается только естественный сплайн ($\tau''_1 = \tau''_{N+1} = 0$), ограничим рассмотрение точками $i = 2, \ldots, N$. Пусть для некоторого набора параметров $p^{(n)}_i$, $i = 1, \ldots, N$, выполняется $\tau''_kb_k < 0$, $k \in [2, N]$. 
Обозначим $$\overline{\lambda} = \frac{\max(|b_k|, (d_{k-1} + d_k) |\tau''_k|)}{2 \max(|\tau''_{k-1}|, |\tau''_{k+1}|)}$$
и положим $\tilde{p_i} = (\overline{\lambda} h_i)^{-1/2}$, $i = k-1, k$ (или возможно $\tilde{p_i} = \max(p^{(n)}_i, (\overline{\lambda} h_i)^{-1/2})$, если необходимо только увеличивать параметр).

Для получения следующих значений параметров натяжения применим механизм релаксации: $$p^{(n+1)} = p^{(n)} + \omega (\tilde{p} - p^{(n)})$$

Данный итеративный процесс позволяет дополнительно \enquote{натянуть} сплайн в интервалах с ложными точками перегиба, при этом он не всегда помогает бороться с ложными экстремумами.

\pagebreak
